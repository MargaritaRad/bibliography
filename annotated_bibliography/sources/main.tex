\documentclass[a4paper,12pt,bold]{scrartcl}

\renewcommand{\baselinestretch}{1.3}\normalsize
\newcommand{\vect}[1]{\mathbf{#1}}
\newcommand{\thin}{\thinspace}
\newcommand{\thick}{\thickspace}
\newcommand{\N}{\mathcal{N}}	%Normal Distribution
\newcommand{\U}{\mathrm{U}}	%Uniform Distribution
\newcommand{\D}{\mathrm{D}}	%Dirichlet Distribution
\newcommand{\W}{\mathrm{W}}	%Wishart Distribution
\newcommand{\E}{\mathrm{E}}		%Expectation
\newcommand{\Iden}{\mathbb{I}}	%Identity Matrix
\newcommand{\Ind}{\mathrm{I}}	%Indicator Function

\newcommand{\bs}{\boldsymbol}
\newcommand{\var}{\mathrm{var}\thin}
\newcommand{\plim}{\mathrm{plim}\thin}
\newcommand{\cov}{\mathrm{cov}\thin}
\newcommand\indep{\protect\mathpalette{\protect\independenT}{\perp}}
\def\independenT#1#2{\mathrel{\rlap{$#1#2$}\mkern5mu{#1#2}}}
\usepackage{bbm}
%\usepackage{endfloat}
\renewcommand{\vec}[1]{\mathbf{#1}}

\parindent0pt
\usepackage{algpseudocode,tabularx,ragged2e}
\newcolumntype{C}{>{\centering\arraybackslash}X} % centered "X" column
\newcolumntype{L}{>{\arraybackslash}X} % centered "X" column

\usepackage{algorithmicx}
\usepackage{graphicx}

\usepackage{algorithm}
\graphicspath{{../codes/}}


\usepackage{apacite}


\usepackage{booktabs}
\usepackage{epigraph}
\usepackage[sans]{dsfont}
\usepackage[round,longnamesfirst]{natbib}
\usepackage{bm}																									%matrix symbol
\usepackage{setspace}																						%Fu�noten (allgm.
\usepackage[colorlinks = true,
            linkcolor = blue,
            urlcolor  = blue,
            citecolor = blue,
            anchorcolor = blue]{hyperref}%Zeilenabst�nde)
\usepackage{threeparttable}
\usepackage{lscape}																							%Querformat
\usepackage[latin1]{inputenc}																		%Umlaute
\usepackage{graphicx}
\usepackage{amsmath}
\usepackage{amssymb}
\usepackage{fancybox}																						%Boxen und Rahmen
\usepackage{appendix}
\usepackage{listings}
\usepackage{xr}

\usepackage{enumerate}
\usepackage[labelfont=bf]{caption}
																		%EURO Symbol
\usepackage{tabularx}
\usepackage{longtable}
\usepackage{subfig,float}																				%Mehrseitige Tabellen
\usepackage{color,colortbl}																			%Farbige Tabellen
\usepackage[left=2cm, right=2cm, top=2cm, bottom=2.5cm]{geometry} %Seitenr�nder
%\usepackage[normal]{caption2}[2002/08/03]												%Titel ohne float - Umgebung
\definecolor{lightgrey}{gray}{0.95}	%Farben mischen
\definecolor{grey}{gray}{0.85}
\definecolor{darkgrey}{gray}{0.80}

\newcommand{\mc}{\multicolumn}

\usepackage{tikz}
\usetikzlibrary{positioning}

\usepackage{caption}
\captionsetup[figure]{labelfont=bf}

\usepackage{url}  % Used for linebreaks in verbatim statements


\usepackage{bibentry}
\nobibliography*

\newtheorem{Definition}{Definition}
\newtheorem{Remark}{Remark}
\newtheorem{Lemma}{Lemma}
\newtheorem{Theorem}{Theorem}
\newtheorem{Excercise}{Excercise}
\newtheorem{Result}{Result}
\newtheorem{Proposition}{Proposition}
\newtheorem{Prediction}{Prediction}
\newtheorem{Solution}{Solution}
\newtheorem{Problem}{Problem}

\setlength{\skip\footins}{1.0cm}
\deffootnote[1em]{1.1em}{0em}{\textsuperscript{\thefootnotemark}}
\renewcommand{\arraystretch}{1.05}

\DeclareMathOperator*{\argmin}{arg\,min}
\DeclareMathOperator*{\argmax}{arg\,max}

\makeatletter
\newenvironment{manquotation}[2][2em]
  {\setlength{\@tempdima}{#1}%
   \def\chapquote@author{#2}%
   \parshape 1 \@tempdima \dimexpr\textwidth-2\@tempdima\relax%
   \itshape}
  {\par\normalfont\hfill--\ \chapquote@author\hspace*{\@tempdima}\par\bigskip}
\makeatother

\newenvironment{boenumerate}
{\begin{enumerate}\renewcommand\labelenumi{\textbf{(\theenumi)}}}
{\end{enumerate}}


\title{Annotated Bibliography}
\author{}
\date{}


\begin{document}
\maketitle
\tableofcontents

%-------------------------------------------------------------------------------
\section{Computational Methods}
%-------------------------------------------------------------------------------
\begin{itemize}
\item \bibentry{Snyman.2005}

The book provides a very nice introduction to some practical issues in optimization. It will be useful to revisit when we tackle computational aspects of structural model estimation.

\end{itemize}
%-------------------------------------------------------------------------------
\section*{Miscellaneous}
%-------------------------------------------------------------------------------
\begin{itemize}
\item \bibentry{Heyman.2003}

This book contains a highly useful exposition of the expected utility theory and its use in operations research in Chapter 2. The chapter is available as a pdf.

\item \bibentry{Smith.2014}

A highly readable introduction to uncertainty quantification.

\item \bibentry{Huber.2009}

This is the foundational textbook for robust statistics. It defines robustness as insensitivity to small deviations from the assumptions. The book is primarily concerned with distributional robustness, i.e. the shape of the true underlying distribution deviates slightly from the assumed model. Another textbook introduction is provided by \citet{Maronna.2006}.

\item \bibentry{Wise.1985a}

This paper is often cited in the context of validation a behavioral model using experimental data. However, the behavioral model itself is just very, very simple.

\item \bibentry{Skiadas.2009}

A textbook introduction to asset pricing, it is very nicely written and in particular Chapter 4 on risk aversion is a very good overview on the topic. Also,  he always includes a lot of notes at the end of each chapter that are a very informative on the ongoing developments and discussions in the literature.

\item \bibentry{Koehn.2017}

In this book the author develops a new approach to uncertainty in economics, which calls for a fundamental change in the methodology of economics. It provides a comprehensive overview and critical appraisal of the economic theory of uncertainty and shows that uncertainty was originally conceptualized both as an epistemic and an ontological problem. As a result of the economic professions' attempt to become acknowledged as a science, the more problematic aspect of ontological uncertainty has been neglected and the subjective probability approach to uncertainty became dominant in economic theory. A careful analysis of ontological theories of uncertainty explains the blindness of modern economics to economic phenomena such as instability, slumps or excessive booms. Based on these findings the author develops a new approach that legitimizes a New Uncertainty Paradigm in economics.

\begin{itemize}
\item useful for numerous non-conventional references on the role of uncertainty in economics
\end{itemize}

\end{itemize}

\bibliography{../../literature}
\bibliographystyle{apalike}

\end{document}
